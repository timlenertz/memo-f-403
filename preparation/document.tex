\documentclass[a4paper,10pt]{scrreprt}
\usepackage[utf8]{inputenc}
\usepackage{graphicx}
\usepackage{mathtools}
\usepackage{gnuplottex}
\usepackage{fullpage}
\usepackage{hyperref}
\usepackage{url}
\usepackage[T1]{fontenc}
\usepackage[frenchb]{babel}

\begin{document}

\title{Recalage et fusion de modèles numérisés tridimensionnels de grande taille}
\subtitle{MEMO-F-403 - Préparation au mémoire}
\author{Tim Lenertz}
\date{\today}
\maketitle

\tableofcontents

\chapter{Introduction}
Pour des projets de documentation 3D, des objets ou environnements sont souvent numérisés sous forme de \emph{nuages de points}. Un nuage de points consiste uniquement en un ensemble de points situés sur les surfaces des objets, définis par des coordonnées dans un repère orthonormal donné, et attribués par des valeurs scalaires ou vectorielles comme par exemple une couleur RGB.

Ces jeux de données sont usuellement capturés par des scanners 3D à télémètre laser, ou par photogrammétrie. Ces données bruts ne représentent qu'une partie de l'objet, et sont limitées par le champ de vision de la caméra, les surfaces cachées (sur le côté opposé au scanner de l'objet), les limites de résolution pour des parties éloignés ou qui ont une texture complexe. Afin d'en synthétiser un nuage de points représentant le modèle complet, plusieurs traitement doivent être effectués, notamment les recalage: Pour toutes les nuages de points partielles, une transformation affine est appliquée aux points afin de les mettre dans un repère commun.

Plusieurs techniques (semi)-automatisées ont été développées qui permettent de trouver une approximation des positions et orientations relatifs des scans. Ils peuvent faire intervenir des photos, ou des marqueurs visuels ajoutés sur la scène. La matrice de transformation précise est ensuite déterminée algorithmiquement, afin de bien aligner les surfaces dans les différents ensembles de points. Typiquement, une forme de l'algorithme ICP\footnote{Iterative Closest Point} est utilisé, un algorithme qui ajuste la transformation en itérativement minimisant la distance entre les deux ensembles de points.

Ce mémoire se concentre sur le recalage de numérisations à distance d'un objet à grande dimensions, avec des numérisations à courtes distances de détails de d'objet. Donc les ensembles de points à recaler pourront avoir des densités de points très différentes, et auront un nombre de points très élevé. Le but est d'établir un workflow et de développer des algorithme qui permettent d'effectuer ce type de recalage. Le travail est basé un projet de documentation 3D actuel du LISA. 


\chapter{Etat de l'art}

\section{Documentation 3D}
Des nuages de points créés via des processus de numérisation 3D sont utilisés dans plusieurs domaines pour représenter des objets réels. Les modèles 3D peuvent passer de représentations détaillées d'objets petits à des bâtiments ou même des sites entiers. Des exemples d'usage sont par exemple la documentation détaillée de sites archéologiques \cite{Web1} \cite{Kein2011} \cite{Grus2012} et d'environnements urbains \cite{Kers2006}, des acquisitions aériennes de terrains, la vision algorithmique en robotique \cite{Bibe2003}, la capture de formes 3D pour création de prothèses en médecine dentaire et orthopédique, la rétroingénierie, la création de modèles en animation 3D.

Pour collectionner les données bruts de l'objet pour lequel on veut générer un modèle 3D, deux techniques sont généralement utilisés: La \emph{photogrammétrie} consiste à prendre des photos (possiblement stéréoscopiques) de l'objet, à partir desquelles on peut extraire algorithmiquement de l'information sur la profondeur des pixels. D'autre part, on utilise des \emph{scanners tridimensionnels}, des appareils qui effectuent un balayage par laser de l'objet et produisent directement un nuage de points. Souvent les deux techniques sont combinées. Cela permet par exemple de compléter un scan laser très détaillé avec de l'information sur les couleurs des points.

Pour passer des données bruts au modèle final envisagé, un nombre important d'étapes de planification et de post-traitement des données doivent être effectués. Par exemple, un projet de documentation archéologique pour lequel des scans 3D et des données photogrammétriques sont utilisés peut passer par les étapes suivantes: \cite{Lerm2009}
\begin{itemize}
	\item Planification et acquisition des données bruts. Les positions à partir desquelles les scans 3D sont pris doivent être planifiés minutieusement en fonction du terrain, de la forme de l'objet et des spécifications du scanner, afin d'assurer une couverture complète de l'objet, avec la résolution et fidélité ciblée. De même pour les position où des photos sont prises.
	\item Filtrage des nuages de points bruts. Le but est d'enlever der erreurs grosses du scanner, comme des points isolés.
	\item Recalage des nuages de points pour les mettre dans un repère commun. Beaucoup méthodes et algorithmes ont été développés pour automatiser un recalage sous différents conditions, ce qui est le sujet de ce mémoire. Dans certains cas des markers sont placés sur la scène avant l'acquisition des scans et photos. Les nuages de points peuvent ensuite être fusionnés par simple concaténation des points.
	\item Maillage des nuages de points, par example par triangulation de Delaunay. Le but ici est d'obtenir un modèle qui reprend l'information sur les surfaces du modèle. Pour des formes complexes, par exemple de la végétation sur un mur, des procédures additionnels d'analyse et de filtrage sont requis.
	\item Par des méthodes photogrammétriques \cite{Tour009} et radiométriques \cite{Giro2013}, les photos sont modifiés et analysés: On corrige des vignettes, distortions et autres erreurs optiques induits par la lentille de la caméra, égalise les valeurs colorimétriques des différentes photos. Par des détecteurs de zones d'intérêt en reconnaissance d'image, on retrouve des markers, ou d'autres points marquants, afin de mettre en correspondance plusieurs photos et d'en extraire la pose de la caméra et des informations de profondeur.
	\item Recalage des photos sur le modèle pour texturer le maillage. Au cas un modèle non-maillé qui reste sous forme de nuage de points, les photos peuvent être utilisés pour coloriser les points. Ensuite la densité des points peut être uniformisée si nécessaire.
	\item Dans certains contextes, par exemple en rétroingénierie ou en architecture, le but est de finir avec un modèle CAD. Des méthodes ont été développées pour segmenter le modèle, et pour reconnaitre des formes ou objets.
\end{itemize}

\subsection{Scanner tridimensionnels}
Les scanners tridimensionnels utilisent des rayonnements, typiquement des lasers, pour physiquement mesurer des informations de profondeur de leur environnements. Dans les dernières années, cette technologie a connu un développement important. Les scanners 3D disposent d'un niveau avancé d'automatisation, et peuvent produire des données de bonne qualité \cite{Grus2012} à des coût relativement abordables.

\subsubsection{Scanners à temps de vol (LIDAR)}
\subsubsection{Scanners par triangulation}
\subsubsection{Scanners par décalage de phase}
\subsubsection{Scanners par lumière structurée}



\section{Recalage}
Afin de créer un modèle complet, on doit fusionner plusieurs scans qui sont prises de différents points de vues et qui sont tous limités par le champ de vision du scanner. Ces différents nuages de points ont tous des repères différents, relatifs à la position et à l'orientation du scanner, et possiblement ayant des échelles différents.

Puisqu'il s'agit toujours de repères cartésiens orthonormals, on peut mettre un nuage de points dans un autre repère en appliquant une matrice de transformation rigide à tous les points du nuage. Une telle transformation consiste en une translation, rotation, et dans certains cas un facteur de redimensionnement. Dès que les nuages de points sont tous dans le même repère, on peut les fusionner par simple concaténation des points. En pratique, des post- et pré-traitements devront être effectués pour ajuster les densités et la distribution des points, corriger des erreurs dans les données bruts comme des points \emph{outliers}, et autres.

Par recalage, on entend le procédé de trouver ces transformations rigides, de manière automatisé ou semi-automatisé. D'une part on a les algorithmes de recalage approximatif, donc le but est d'analyser le contenu des nuages de points et d'en identifier des régions qui représentent la même partie du modèle afin de trouver un recalage approximatif. Ensuite, ce recalage initial peut être raffiné par un algorithme de recalage précis pour bien aligner les surfaces des nuages de points. Généralement ICP est utilisé pour cela. Sachant que les nuages de points sont déjà alignés de manière à ce que les points qui représentent les mêmes parties du modèle sont proches l'un de l'autre, ICP procède en minimisant les distances entre ces points.

On peut distinguer entre les algorithmes locaux (séquentiels) qui opèrent sur deux nuages de points à la fois, et les algorithmes globaux (simultanés) qui effectuent un recalage de plusieurs nuages de points. Cette dernière méthode est plus complexe, mais a l'avantage qu'il n'y a pas d'erreur de recalage qui s'accumule lors des recalages deux-à-deux. Ces algorithmes tentent au lieu de distribuer l'erreur uniformément parmi les nuages.

La méthode préférée pour déterminer un recalage approximatif dépend surtout du type et de la forme de l'objet numérisé, et des données bruts dont on dispose. Certains algorithmes procèdent en identifiant des droites \cite{Lich2011} ou des plans \cite{Dold2006} dans le modèle, et sont donc plus appropriés pour par exemple des façades simples de bâtiments. Si les scans ont des zones de chevauchement assez grandes, il peut être possible d'en déduire le recalage de façon complètement automatisée. D'autres méthodes peuvent être utilisées avec des données collectionnées additionnelles, comme des photos prises par le même point de vue que le scanner \cite{Tour2009} ou des markers placés sur l'objet avant la numérisation \cite{Mati2011}. Par des algorithmes de détection de zones d'intérêt \cite{Tuyt2007} sur des images ou directement sur les nuages de points, on peut alors identifier des points communs dans les différents scans. Aussi les information attribués aux points, comme une couleur, une mesure de température, ou autres, peuvent servir en tant que dimension en plus des coordonnées 3D pour déterminer leur distance. Une première estimation de la pose des scans peut être issue de sensors odométriques ou GPS du scanner.

Certains algorithmes seront décrit en détail dans les pages suivantes, à savoir ICP \cite{Besl1992} et ses variantes \cite{Rusi2001}, ...




\subsection{Iterative Closest Point}
L'algorithme ICP (``itératif point le plus proche'') est considéré comme la méthode standard pour l'alignement précis de deux nuages de points, après qu'un recalage approximatif initial a déjà été fait. Parmi les deux nuages de points pris comme entrée, l'un (\emph{référence}) reste fixé, tandis que l'autre (\emph{source}) est transformé par translation et rotation. L'algorithme procède en itérativement raffinant la transformation à chaque étape, de manière à ce que la distance entre les deux nuages de points soit minimisée.

Plus précisément, pour tout point $s_i$ de l'ensemble \emph{source}, on choisit un point $c_i$ de l'ensemble \emph{cible} pour lequel on estime qu'il représente plus au moins la même position dans le nuage de points. Comme les deux nuages de points sont déjà recalés approximativement, une méthode courante est de choisir le point le plus proche, avec la transformation actuelle. L'algorithme nécessite donc que les densités de points des deux ensembles sont similaires. Ensuite une transformation est appliquée aux points $s_i$ qui minimise une métrique donnée de l'erreur de recalage.

La \emph{cible} ne doit pas forcément être donnée sous forme de nuage de points, mais peut par exemple être une surface mathématique continue donnée sous forme paramétrique $(x(\vec{t}),y(\vec{t}),z(\vec{t}))$ ou implicite $g(x, y, z) = 0$. On peut alors calculer numériquement les coordonnées d'un point $c_i$ correspondant à chaque point $s_i$. \cite{Besl1992}

Plusieurs variantes de l'algorithme ont été développées qui se distinguent par le choix des couples de points, des poids associés aux couples, le rejet de certains couples, et la façon comment la métrique d'erreur est définie et minimisée. \cite{Rusi2001}

Pour la minimisation, on doit calculer une transformation rigide (généralement translation et rotation, donc 6 degrés de liberté) transforme les points $s_i$. Il existe plusieurs méthodes déterministes qui nécessitent pas de prendre plusieurs échantillons dans l'espace des transformations rigides. Pour le calcul de la rotation on peut par exemple utiliser la décomposition en valeurs singulières, par une représentation sous forme de matrice orthonormale ou sous forme de quaternion \cite{Horn1986}.

En général, un algorithme ICP procède selon les étapes suivantes:
\begin{enumerate}
	\item Sélectionner des sous-ensemble de points ${s_i}$ et ${c_i}$ à utiliser (possiblement tous les points).
	\item Former les couples $(s_i, c_i)$.
	\item Associer des poids aux couples. (optionnel)
	\item Rejeter certains couples. (optionnel)
	\item Calculer une métrique d'erreur à partir des couples. Par exemple $\sum_{i} \|\ s_i - c_i \|^2$.
	\item Appliquer une transformation rigide aux points ${s_i}$ qui minimise l'erreur.
	\item Si l'erreur est inférieur à une valeur de seuil donnée, arrêter. Sinon, continuer avec 1 ou 2. 
\end{enumerate}


\subsubsection{Calcul de la transformation par quaternions}
Une manière de calculer la translation et rotation qui est appliquée à chaque itération à l'ensemble de points \emph{source} est décrite dans \cite{Besl1992} et \cite{Horn1986}. La procédure prend uniquement les $n$ couples de points $(\vec{s}_i, \vec{c}_i)$ comme entrée, où $\vec{c}_i$ est un point de l'ensemble \emph{cible} $C$, et $\vec{s}_i$ un point de l'ensemble \emph{source} $S$. On suppose que les transformations données par les itérations précédentes ont déjà été appliquées aux points $s_i$. La méthode reste valable pour tout choix de points, et $C$ ne doit pas forcément être donné sous forme d'ensemble de points, mais ses points $\vec{c}_i$ peuvent aussi par exemple être calculés à partir d'une surface paramétrique.

Le but de trouver $\vec{q}_T$ et $\dot{q}_R$ qui minimisent le carré des distances des couples $f(\vec{q})$.
\begin{equation}
	f(\vec{q}) = \frac{1}{n} \sum_{i=1}^{n} \| \vec{c}_i - \mathbf{R} \vec{s}_i - \vec{q}_T \|^2
\end{equation}
$\vec{q}_T = (q_4 q_5 q_6)^T$ donne la translation, et $\mathbf{R}$ la matrice de rotation 3x3.

On cherche une transformation de la forme $\vec{t'} = \mathbf{R}(\vec{s}) + \vec{q}_T$, donc une rotation, suivie par une translation à appliquer aux points $s_i$, qui minimise les distances des points transformés $t_i$ aux points $c_i$ correspondants dans le nuage de points \emph{cible}. Car on ne peut pas assumer que les deux nuages $S$ et $C$ auront exactement la même distribution de points, les erreurs $e_i = \vec{c}_i - \vec{t}_i$ ne peuvent en général pas devenir $0$. On minimise la somme des carrés des erreurs $\sum \| e_i \|^2$. Le développement décrit en \cite{Horn1986} inclut aussi un facteur de redimensionnement, mais pour ICP on suppose généralement que les deux nuages de points ont déjà la même échelle. Un redimensionnement durant les itérations introduirait des problèmes avec le choix des couples de points.

Pour le développement suivant, on prend les coordonnées des points $\vec_{s}$ et $\vec_{c}$, relatif au centres de masse des ensembles. On définit
\begin{equation}
	\vec{\mu_s} = \frac{1}{n} \sum_{i} \vec{s}_i
	\hspace{1cm} \text{et} \hspace{1cm}
	\vec{\mu_c} = \frac{1}{n} \sum_{i} \vec{c}_i
\end{equation}

Et pour tout $i$, on pose $\vec{s'}_i = \vec{s}_i - \vec{\mu_s}$ et $\vec{c'}_i = \vec{c}_i - \vec{\mu_c}$. Donc $\sum \vec{s'}_i = 0$ et $\sum \vec{c'}_i = 0$. L'erreur $e_i$ peut être réécrit comme $e_i = \vec{c'}_i - \vec{t'}_i = \vec{c'}_i - \mathbf{R}(\vec{s'}_i) - \vec{q'}_T$. On peut montrer que la somme des carrés des erreur sera minimale quand $\vec{q'}_T = 0$, et donc la translation à appliquer correspondra à $\vec{q}_T = \vec{q}_T - \mathbf{R}(\vec{\mu_s})$:

En fait, on a que $\vec{q'}_T = \vec{q}_T - \vec{\mu_c} + \mathbf{R}(\vec{\mu_s})$. La somme des carrés peut être développée en
\begin{equation}
	\sum_{i=1}^{n} \| \vec{c'}_i - \mathbf{R}(\vec{s'}_i) \|^2 - 2 \vec{q'}_T \sum_{i=1}^{n} (\vec{c'}_i - \mathbf{R}(\vec{s'}_i)) + n \|\vec{q'}_T\|^2
\end{equation}

Le second terme sera toujours $0$ puisque les coordonnées sont pris relatif aux centres de masse. Le premier ne dépend pas de $\vec{q'}_T$. Donc l'expression est minimisée par $\vec{q'}_T = 0 \iff \vec{q}_T = \vec{\mu_c} - \mathbf{R}(\vec{\mu_s})$.

Pour trouver la rotation \textbf{R}, on maximise
\begin{equation}
	\sum_{i=1}^{n} \vec{c'}_{i} \cdot \mathbf{R}(\vec{s'}_i)
\end{equation}
En effet, le produit scalaire de deux vecteurs est maximal quand ils ont la même direction. On cherche la rotation sous forme d'un quaternion $\dot{q}_R$.
\footnote{Une matrice de rotation \textbf{R} peut en être déduit à partir d'un quaternion $\dot{q} = (q_0 q_1 q_2 q_3)^T$ par la formule
\begin{equation}
	\mathbf{R} = \begin{bmatrix}
		q_0^2 + q_1^2 - q_2^2 - q_3^2 & 2(q_1 q_2 - q_0 q_3) & 2(q_1 q_3 + q_0 q_2) \\
		2(q_1 q_2 + q_0 q_3) & q_0^2 + q_2^2 - q_1^2 - q_3^2 & 2(q_2 q_3 - q_0 q_1) \\
		2(q_1 q_3 - q_0 q_2) & 2(q_2 q_3 + q_0 q_1) & q_0^2 + q_3^2 - q_1^2 - q_2^2
	\end{bmatrix}
\end{equation}}

Une rotation peut être encodée dans un quaternion unitaire, c'est à dire pour lequel $q_0 \geq 0$ et $q_0^2 + q_1^2 + q_2^2 + q_3^2 = 1$. $\dot{q}$ est purement imaginaire si $q_0 = 0$. (Les trois autres composants peuvent être considérés comme termes imaginaires comme chez les nombres complexes.) On représente un vecteur sous forme de quaternion purement imaginaire, en mettant ses coordonnées dans $q_1, q_2, q_3$. On peut montrer que si $\dot{s}$ est purement imaginaire et $\dot{q}_R$ est unitaire, alors $\dot{q}_R \, \dot{s} \, \dot{q}^*_R$ reste purement imaginaire, et peut donc représenter une opération sur un vecteur. $\dot{q}^*_R$ est le conjugué de $\dot{q}_R$.

L'article \cite{Horn1986} montre qu'en utilisant les propriétés des quaternions, on peut former à partir des ensembles de points $\vec{s}_i$ et $\vec{c}_i$ une matrice symétrique 4x4 \textbf{N} telle que
\begin{equation}
	\sum_{i=1}^{n} \vec{c'}_{i} \cdot \mathbf{R}(\vec{s'}_i) =
	\sum_{i=1}^{n} \dot{c'}_{i} \cdot (\dot{q}_R \, \dot{s'}_{i} \, \dot{q}^*_R) =
	\dot{q}_R \, \mathbf{N} \, \dot{q}^*_R
\end{equation}
Il est aussi montré que le quaternion unitaire $\dot{q}_R$ qui maximise l'expression est le vecteur propre correspondant à la valeur propre la plus grande de \textbf{N}. Le calcul des valeurs propres nécessite la résolution d'une equation polynomiale du quatrième degré. Il existe des solutions de forme fermée pour ce problème.

Donc, cette méthode permet de déterminer la rotation et translation à appliquer à \textbf{M} par une série de calculs, sans passer par des méthodes approximatives.


\subsubsection{Variantes de l'algorithme}



\subsubsection{Version simultanée}
Une approche qui permet de faire un recalage ICP \emph{simultané} de plusieurs nuages de points est décrite dans \cite{Told2010}.

\subsection{Extended Gaussian Images}
On peut générer à partir d'un nuage de point son EGI (Extended Gaussian Image), qui est une forme d'encoder la forme d'un objet convexe. Ceci permet d'estimer la rotation à appliquer, sans partir d'un recalage initial.


\bibliographystyle{plain}
\bibliography{../references}

\end{document}
